\documentclass[ignorenonframetext,]{beamer}
\setbeamertemplate{caption}[numbered]
\setbeamertemplate{caption label separator}{: }
\setbeamercolor{caption name}{fg=normal text.fg}
\beamertemplatenavigationsymbolsempty
\usepackage{lmodern}
\usepackage{amssymb,amsmath}
\usepackage{ifxetex,ifluatex}
\usepackage{fixltx2e} % provides \textsubscript
\ifnum 0\ifxetex 1\fi\ifluatex 1\fi=0 % if pdftex
  \usepackage[T1]{fontenc}
  \usepackage[utf8]{inputenc}
\else % if luatex or xelatex
  \ifxetex
    \usepackage{mathspec}
  \else
    \usepackage{fontspec}
  \fi
  \defaultfontfeatures{Ligatures=TeX,Scale=MatchLowercase}
\fi
% use upquote if available, for straight quotes in verbatim environments
\IfFileExists{upquote.sty}{\usepackage{upquote}}{}
% use microtype if available
\IfFileExists{microtype.sty}{%
\usepackage{microtype}
\UseMicrotypeSet[protrusion]{basicmath} % disable protrusion for tt fonts
}{}
\newif\ifbibliography
\hypersetup{
            pdfborder={0 0 0},
            breaklinks=true}
\usepackage{color}
\usepackage{fancyvrb}
\newcommand{\VerbBar}{|}
\newcommand{\VERB}{\Verb[commandchars=\\\{\}]}
\DefineVerbatimEnvironment{Highlighting}{Verbatim}{commandchars=\\\{\}}
% Add ',fontsize=\small' for more characters per line
\usepackage{framed}
\definecolor{shadecolor}{RGB}{248,248,248}
\newenvironment{Shaded}{\begin{snugshade}}{\end{snugshade}}
\newcommand{\KeywordTok}[1]{\textcolor[rgb]{0.13,0.29,0.53}{\textbf{{#1}}}}
\newcommand{\DataTypeTok}[1]{\textcolor[rgb]{0.13,0.29,0.53}{{#1}}}
\newcommand{\DecValTok}[1]{\textcolor[rgb]{0.00,0.00,0.81}{{#1}}}
\newcommand{\BaseNTok}[1]{\textcolor[rgb]{0.00,0.00,0.81}{{#1}}}
\newcommand{\FloatTok}[1]{\textcolor[rgb]{0.00,0.00,0.81}{{#1}}}
\newcommand{\ConstantTok}[1]{\textcolor[rgb]{0.00,0.00,0.00}{{#1}}}
\newcommand{\CharTok}[1]{\textcolor[rgb]{0.31,0.60,0.02}{{#1}}}
\newcommand{\SpecialCharTok}[1]{\textcolor[rgb]{0.00,0.00,0.00}{{#1}}}
\newcommand{\StringTok}[1]{\textcolor[rgb]{0.31,0.60,0.02}{{#1}}}
\newcommand{\VerbatimStringTok}[1]{\textcolor[rgb]{0.31,0.60,0.02}{{#1}}}
\newcommand{\SpecialStringTok}[1]{\textcolor[rgb]{0.31,0.60,0.02}{{#1}}}
\newcommand{\ImportTok}[1]{{#1}}
\newcommand{\CommentTok}[1]{\textcolor[rgb]{0.56,0.35,0.01}{\textit{{#1}}}}
\newcommand{\DocumentationTok}[1]{\textcolor[rgb]{0.56,0.35,0.01}{\textbf{\textit{{#1}}}}}
\newcommand{\AnnotationTok}[1]{\textcolor[rgb]{0.56,0.35,0.01}{\textbf{\textit{{#1}}}}}
\newcommand{\CommentVarTok}[1]{\textcolor[rgb]{0.56,0.35,0.01}{\textbf{\textit{{#1}}}}}
\newcommand{\OtherTok}[1]{\textcolor[rgb]{0.56,0.35,0.01}{{#1}}}
\newcommand{\FunctionTok}[1]{\textcolor[rgb]{0.00,0.00,0.00}{{#1}}}
\newcommand{\VariableTok}[1]{\textcolor[rgb]{0.00,0.00,0.00}{{#1}}}
\newcommand{\ControlFlowTok}[1]{\textcolor[rgb]{0.13,0.29,0.53}{\textbf{{#1}}}}
\newcommand{\OperatorTok}[1]{\textcolor[rgb]{0.81,0.36,0.00}{\textbf{{#1}}}}
\newcommand{\BuiltInTok}[1]{{#1}}
\newcommand{\ExtensionTok}[1]{{#1}}
\newcommand{\PreprocessorTok}[1]{\textcolor[rgb]{0.56,0.35,0.01}{\textit{{#1}}}}
\newcommand{\AttributeTok}[1]{\textcolor[rgb]{0.77,0.63,0.00}{{#1}}}
\newcommand{\RegionMarkerTok}[1]{{#1}}
\newcommand{\InformationTok}[1]{\textcolor[rgb]{0.56,0.35,0.01}{\textbf{\textit{{#1}}}}}
\newcommand{\WarningTok}[1]{\textcolor[rgb]{0.56,0.35,0.01}{\textbf{\textit{{#1}}}}}
\newcommand{\AlertTok}[1]{\textcolor[rgb]{0.94,0.16,0.16}{{#1}}}
\newcommand{\ErrorTok}[1]{\textcolor[rgb]{0.64,0.00,0.00}{\textbf{{#1}}}}
\newcommand{\NormalTok}[1]{{#1}}
\usepackage{graphicx,grffile}
\makeatletter
\def\maxwidth{\ifdim\Gin@nat@width>\linewidth\linewidth\else\Gin@nat@width\fi}
\def\maxheight{\ifdim\Gin@nat@height>\textheight0.8\textheight\else\Gin@nat@height\fi}
\makeatother
% Scale images if necessary, so that they will not overflow the page
% margins by default, and it is still possible to overwrite the defaults
% using explicit options in \includegraphics[width, height, ...]{}
\setkeys{Gin}{width=\maxwidth,height=\maxheight,keepaspectratio}

% Prevent slide breaks in the middle of a paragraph:
\widowpenalties 1 10000
\raggedbottom

\AtBeginPart{
  \let\insertpartnumber\relax
  \let\partname\relax
  \frame{\partpage}
}
\AtBeginSection{
  \ifbibliography
  \else
    \let\insertsectionnumber\relax
    \let\sectionname\relax
    \frame{\sectionpage}
  \fi
}
\AtBeginSubsection{
  \let\insertsubsectionnumber\relax
  \let\subsectionname\relax
  \frame{\subsectionpage}
}

\setlength{\parindent}{0pt}
\setlength{\parskip}{6pt plus 2pt minus 1pt}
\setlength{\emergencystretch}{3em}  % prevent overfull lines
\providecommand{\tightlist}{%
  \setlength{\itemsep}{0pt}\setlength{\parskip}{0pt}}
\setcounter{secnumdepth}{0}
\usepackage[british]{babel}
\usepackage{graphicx,hyperref,anderson,url}
\usepackage{fontawesome}

\title[SEES Update]{Update-- NSF SEES}
\subtitle{Annual Meeting, NSF SEES Investigators}
\date{November 17, 2017}

\author[Brooke Anderson]{
  Brooke Anderson \\\medskip
  {\small \faEnvelope: \url{brooke.anderson@colostate.edu}} \\
  {\small \faGithub:  \url{www.github.com/geanders}}}

\institute[Colorado State University]{
  Department of Environmental \& Radiological Health Sciences \\
  Environmental Epidemiology Section \\
  Colorado State University}

\date{}

\begin{document}

\begin{frame}
  \titlepage
\end{frame}

\section{Projecting heat wave-related mortality
impacts}\label{projecting-heat-wave-related-mortality-impacts}

\begin{frame}{Historical heat waves}

Variation across heat waves in increased mortality risk.

\vspace{-0.2in}

\begin{center}\includegraphics[width=\textwidth]{anderson_sees_2017_files/figure-beamer/unnamed-chunk-11-1} \end{center}

\vspace{-0.2in}

\footnotesize

The red line shows the central estimate across all heat waves. The graph
shows estimates from 2,980 heat waves identified in 83 U.S. communities,
1987-2005.

\end{frame}

\begin{frame}{Health-based model}

\begin{block}{Model goal}
Predict heat wave-related mortality for a heat wave based on a number of its characteristics.
\end{block}

\small

\begin{block}{Model development}
\begin{enumerate}
\item Identified historical heat waves in 83 U.S. communities, 1987 to 2005. (Heat wave = $\ge2$ days $\ge98^{th}$ percentile temperature.)
\item Used epidemiological model to estimate relative risk of mortality observed for 2,980 historical heat waves.
\item Built a Random Forests model to predict the relative risk of a heat wave based on 20 heat wave characteristics.
\item Validated model through cross-validation, tuned, compared to other models.
\end{enumerate}
\end{block}

\end{frame}

\begin{frame}{Heat wave characteristics}

\begin{center}\includegraphics[height=0.85\textheight]{anderson_sees_2017_files/figure-beamer/unnamed-chunk-12-1} \end{center}

\end{frame}

\begin{frame}{Variable importance for heat-health model}

\begin{center}\includegraphics[height=0.85\textheight]{anderson_sees_2017_files/figure-beamer/unnamed-chunk-13-1} \end{center}

\end{frame}

\begin{frame}[plain]

\begin{center}\includegraphics[height=1.1\textheight]{HeatwaveCalendar} \end{center}

\end{frame}

\begin{frame}{Projected excess heat wave mortality}

\includegraphics{anderson_sees_2017_files/figure-beamer/unnamed-chunk-15-1.pdf}

\vspace{-0.1in}

\footnotesize

Example from one ensemble member of the \(1.5^{o}C\) model output,
showing characteristics and projected mortality for heat waves in ten
larger U.S. communities. The dotted line shows where points would fall
for heat waves with a 10\% increase in daily mortality.

\end{frame}

\begin{frame}{Incorporating A/C effect modification}

\begin{center}\includegraphics[width=7.51in,height=0.85\textheight]{hazard_sees_overview_update} \end{center}

\end{frame}

\section{Tropical storm exposure in U.S.
counties}\label{tropical-storm-exposure-in-u.s.-counties}

\begin{frame}{Hazard-specific tropical storm metrics}

\begin{columns}
\begin{column}{0.5\textwidth}
\begin{block}{Tropical storm hazard metrics}
   \begin{itemize}
    \item Distance from the storm
    \item High winds
    \item Rainfall
    \item Storm surge
    \item Flood events
    \item Tornado events
   \end{itemize}
\end{block}
\end{column}
\begin{column}{0.5\textwidth}  
    \vspace{-0.25cm}
    \begin{center}
     \includegraphics[width=0.8\textwidth]{storm_hazards.png}
     \end{center}
     \vspace{-0.25cm}
     \scriptsize{Image sources: Los Angeles Times, NBC}
\end{column}
\end{columns}

\end{frame}

\begin{frame}{Assessing tropical storm exposure}

\begin{block}{Challenge for epidemiological research}
How to determine whether a county was exposed to a tropical storm?
\end{block}

\vspace{-0.3cm}

\begin{center}\includegraphics[height=0.77\textheight]{previous_exposure_metrics} \end{center}

\end{frame}

\begin{frame}{Project aims}

\begin{block}{Work on tropical storm exposure}
\begin{itemize}
  \item Develop exposure classifications of all U.S. Atlantic basin tropical storms, 1996--2011, based on reasonable measurements of tropical storm hazards
  \item Assess agreement between hazard-based county-specific exposure classifications
  \item Make exposure assessments accessible to other researchers for epidemiological and other impact studies 
\end{itemize}
\end{block}

\end{frame}

\begin{frame}{Assessing tropical storm exposure}

\begin{columns}
\begin{column}{0.5\textwidth}

\includegraphics[width=\textwidth]{finding_closest_point} 

\vspace{-0.5cm}
\small
Example of "Best Tracks" data
\end{column}
\begin{column}{0.5\textwidth}
\small
\begin{block}{Distance metric}
\begin{itemize}
\item \textbf{Distance:} National Hurricane Center Best Tracks data
\item \textbf{Wind:} Wind model based on Willoughby et al. (2006)
\item \textbf{Rain:} Re-analysis rain data (NLDAS-2)
\item \textbf{Flood and tornado events:} NOAA Storm Events database
\end{itemize}
\end{block}
\end{column}
\end{columns}

\end{frame}

\section{Agreement between exposure
metrics}\label{agreement-between-exposure-metrics}

\begin{frame}{County-level exposure to Hurricane Ivan (2004)}

\vspace{-0.6cm}

\begin{center}\includegraphics[height=0.75\textheight]{ivanexposurepresentation} \end{center}

\vspace{-0.5cm} \scriptsize Criteria for exposure classifications:
\textbf{Distance:} Within 100 kms of storm track. \textbf{Rain:} \(\ge\)
75 mm of rain total for two days before to one day after storm.
\textbf{Wind:} Modeled wind of \(\ge\) 15 m/s. \textbf{Flood, Tornado:}
Listed event in NOAA Storm Events database.

\end{frame}

\begin{frame}{County-level agreement in storm exposure}

\begin{block}{Assessing agreement in county classifications}
For each storm and each pair of metrics, we measured the \textit{Jaccard index} as a measure of county-level agreement in exposure classification for a storm:

\begin{equation*}
J = \frac{X_1 \cap X_2}{X_1 \cup X_2}
\end{equation*}

where $X_1$ is the set of counties exposed to a storm based on the first metric and $X_2$ is the set of counties exposed to the storm based on the second metric. 

\end{block}

\end{frame}

\begin{frame}{County-level agreement in storm exposure}

\vspace{-0.3cm}

\begin{center}\includegraphics[height=0.87\textheight]{jaccard_heatmap_presentation} \end{center}

\end{frame}

\begin{frame}{Tropical storm exposure in U.S. counties}

\begin{centering}
\small Storm hits per county per decade based on rain (left) and wind (right) exposure metrics.
\end{centering}

\begin{center}\includegraphics[width=0.95\textwidth]{hurricane_exposure} \end{center}

\vspace{-0.7cm} \scriptsize Criteria for exposure classifications:
\textbf{Rain:} \(\ge\) 75 mm of rain total for two days before to one
day after storm. \textbf{Wind:} Modeled wind of \(\ge\) 15 m/s.

\end{frame}

\begin{frame}[plain]

\begin{center}\includegraphics[width=15.62in,height=1.1\textheight]{tropical_storm_rrs} \end{center}

\end{frame}

\section{Software}\label{software}

\begin{frame}[fragile]{Project software}

\footnotesize

\begin{block}{`hurricaneexposure`}
Create county-level exposure time series for tropical storms in U.S. counties. Exposure can be determined based on several hazards (e.g., distance, wind, rain), with user-specified thresholds. On CRAN.
\end{block}

\begin{Shaded}
\begin{Highlighting}[]
\KeywordTok{county_rain}\NormalTok{(}\DataTypeTok{counties =} \KeywordTok{c}\NormalTok{(}\StringTok{"22071"}\NormalTok{, }\StringTok{"51700"}\NormalTok{), }\DataTypeTok{rain_limit =} \DecValTok{100}\NormalTok{, }
            \DataTypeTok{start_year =} \DecValTok{1995}\NormalTok{, }\DataTypeTok{end_year =} \DecValTok{2005}\NormalTok{, }\DataTypeTok{dist_limit =} \DecValTok{100}\NormalTok{,}
            \DataTypeTok{days_included =} \KeywordTok{c}\NormalTok{(-}\DecValTok{1}\NormalTok{, }\DecValTok{0}\NormalTok{, }\DecValTok{1}\NormalTok{))}
\end{Highlighting}
\end{Shaded}

\begin{verbatim}
## # A tibble: 4 x 5
##       storm_id  fips closest_date storm_dist tot_precip
##          <chr> <chr>        <chr>      <dbl>      <dbl>
## 1    Bill-2003 22071   2003-06-30   38.78412      141.1
## 2 Charley-2004 51700   2004-08-14   43.01152      136.2
## 3   Cindy-2005 22071   2005-07-06   32.21758      113.2
## 4   Floyd-1999 51700   1999-09-16   46.50729      207.5
\end{verbatim}

\end{frame}

\begin{frame}{Project software}

\begin{columns}
\begin{column}{0.3\textwidth}
\footnotesize
\begin{block}{`stormwindmodel`}
Model storm winds from Best Tracks data at U.S. locations. Includes modeling sustained and gust winds, as well as duration of sustained and gust winds above a specified threshold. On CRAN.
\end{block}
\end{column}
\begin{column}{0.7\textwidth}

\begin{center}\includegraphics[width=\textwidth]{census_track_modeled_winds} \end{center}
\end{column}
\end{columns}

\end{frame}

\begin{frame}{Project software}

\footnotesize

\begin{block}{`countyweather`, `countyfloods`}
Download weather monitor data through NOAA and USGS APIs by U.S. county. Includes functions to map available monitors / gages for each county. On CRAN.
\end{block}

\footnotesize

\begin{block}{`noaastormevents`}
Download and explore listings from the NOAA Storm Events database. Includes the ability to pull events based on a tropical storm, using events listed close in time and distance to the storm's tracks. On CRAN.
\end{block}

\footnotesize

\begin{block}{`countytimezones`}
Convert time-stamps from UTC to local time zones for U.S. counties based on county FIPs. Facilitates merging weather observations with locally measured data, including health outcomes. On CRAN.
\end{block}

\end{frame}

\end{document}
